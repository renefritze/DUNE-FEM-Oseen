The complete documentation of \Fem\ is done using the tool \texttt{doxygen}.
You can find this documentation online under \url{http://www.mathematik.uni-freiburg.de/IAM/Research/projectskr/dune/doc/html/index.html}.
You can also build your own local \texttt{doxygen} documentation, by removing the \texttt{--disable-documentation} command in your \texttt{config.opts} file (see listing on page \pageref{config.opts})

Some hints about using the \texttt{doxygen} documentation:
\begin{itemize}
\item The best way to start is from the page Modules (\url{http://www.mathematik.uni-freiburg.de/IAM/Research/projectskr/dune/doc/html/modules.html}) which gives you access to the documentation by category.

\item A list of the central interface classes can be found here: \url{http://www.mathematik.uni-freiburg.de/IAM/Research/projectskr/dune/doc/html/interfaceclass.html}.

\item A summary of the main features and concepts can be found here: \url{http://www.mathematik.uni-freiburg.de/IAM/Research/projectskr/dune/doc/html/group__FEM.html}.

\item Newly added implementations are linked on this page: \url{http://www.mathematik.uni-freiburg.de/IAM/Research/projectskr/dune/doc/html/newimplementation.html}.

\item Some remarks about writting dune-fem documentation is found here: \url{http://www.mathematik.uni-freiburg.de/IAM/Research/projectskr/dune/doc/html/DocRules.html}.

\item And finally some notes on using subversion (svn):
\url{http://www.mathematik.uni-freiburg.de/IAM/Research/projectskr/dune/doc/html/svnhelp.html}
\end{itemize}
