\chapter{Introduction}

  The implementation of a discrete model of a partial differential equation in \Fem\ is based on the
  general concept of
    \begin{itemize}
    \item function spaces, 
    \item functions and
    \item operators, that act on functions.
  \end{itemize}

%%%%%%%%%%%%%%%%%%%%%%%%%%%%%%%%%%%%%%%%%%%%%%%%%%%%%%%%%%%%%%%%

\chapter{Functions and FunctionSpaces}

%%%%%%%%%%%%%%%%%%%%%%%%%%%%%%%%%%%%%%%%%%%%%%%%%%%%%%%%%%%%%%%%

\section{Abstract definition of function spaces and functions}

  A function space $V$ in our concept is a set of mappings from the domain $D:=\mathbb{K}_D^d$ to the range $R:=\mathbb{K}_R^n$, e.g.
  \begin{eqnarray}
    V := \{ u : \mathbb{K}_D^d \to \mathbb{K}_R^n \} \nonumber 
  \end{eqnarray}
  Here, $\mathbb{K}_D$ denotes the domain field, $\mathbb{K}_R$ the range field and  $d,n$ the dimensions of the domain and range, respectively. To further specify the function space, additional properties can be added, e.g. the function are in $C^m$ or do belong to the Sobolev Space $H^m$. \index{function space} \index{domain} \index{range} \index{domain field} \index{range field}

  A discrete function space \index{discrete function space} $V_h$ with finite dimension $m$ is a subset of a function space with the property that the functions are defined locally on the elements $e$ of the underlying computational grid $\cal T$. If $\hat e $ denotes the reference element of $e$ and $F_e$ the mapping $F_e: \hat e \to e$ (for further details refer to \citep{GitterPapier:06}), we define the local base function set $V_{\hat e}$ on the reference element $\hat e$ through
  \begin{eqnarray}
    V_{\hat e} := \{\varphi_1, \ldots, \varphi_{{\rm dim}(V_{\hat e})}\}. \nonumber 
  \end{eqnarray}
  The discrete function space $V_h$ is then given as 
\begin{eqnarray}
    V_h := \Big\{ u_h \in V: u_h|_e := u_e := \sum_{\varphi \in V_{\hat e}} g(u_{e,\varphi}) \ \varphi \circ F_e^{-1}, \ {\rm for \ all } \ e \in {\cal T}\big\}. \nonumber 
\end{eqnarray}
  We call $V_e := {\rm span}\{\varphi \circ F_e^{-1}: \varphi \in V_{\hat e}\}$ a local function space, $u_e \in V_e$ a local function, and ${\rm DOF}_e := \{u_{e,\varphi}, \varphi \in V_{\hat e}\}$ the set of local degrees of freedom. In oder to incorporate global properties of the discrete function space, the function space has to provide a mapping $g$ between the local degrees of freedom (${\rm DOF}_e$) and the global degrees of freedom ${\rm DOF} := \{ u_i: i = 0, \cdots, m\}$. \index{function space!local} \index{DOF, degree of freedom}

  We summarize, that a discrete function space $V_h$ is determined by a function space $V$, a grid $\cal T$, the base function sets $V_{\hat e}$ for all reference elements $\hat e$ and the mapping $g$ from local to global degrees of freedom. A discrete function $u_h \in V_h$ is accordingly defined as a set of local functions $u_e$ where a local function provides access to the local degrees of freedom (${\rm DOF}_e$).


  \section{FunctionSpaces and Mappings}

   A function space $V$ in our concept is a set of mappings from the domain $D:=\mathbb{K}_D^d$ to the range $R:=\mathbb{K}_R^n$, e.g.

  \begin{eqnarray}
    V := \{ u : \mathbb{K}_D^d \to \mathbb{K}_R^n \} \nonumber 
  \end{eqnarray}

  Here, $\mathbb{K}_D$ denotes the domain field, $\mathbb{K}_R$ the range field and  $d,n$ the dimensions of the domain and range, respectively. 


  \section{Grids and GridParts}

  In this section we will remember how to create grids and GridParts. The abstract definition of \Dune\ grids can be found  in \citep{GitterPapier:06}. 

  Remember, a hierarchic grid $\Th$ with $L+1$ levels consists of $L+1$ grids 
  $$\Th := \left\{\mathcal{T}_0, \mathcal{T}_1, \ldots \mathcal{T}_L\right\},$$
  and $\mathcal{T}_i$ is formed up by the codim 0 entities of level $i$, $i=0,\ldots,L$. 

  We call $\mathcal{T}_0$ the macro grid \index{grid!macro}. Another important grid consisting of codim 0 entities of a \Dune\ grid is the leaf grid \index{grid!leaf} $\mathcal{T}_{leaf}$. It is made up by the codim 0 entities that have no children.  

  Each of these grids, $\mathcal{T}_0, \ldots, \mathcal{T}_L, \mathcal{T}_{leaf}$, is suitable for computations within the DUNE FEM context! With the GridPart class you can choose one these grids. This is necessary for the definition of a function space. Imagine a GridPart as a view on a part of the underlying grid. Most of the methods of the GridPart class then replace the methods offered by the grid e.g. when working with the discrete functions. The interface methods are
 

\section{FunctionSpaces}

  In this section we will show all the defintions needed to define a function space and a function and how to evaluate function.

%%%%%%%%%%%%%%%%%%%%%%%%%%%%%%%%%%%%%%%%%%%%%%%%%%%%%%%%%%%%%%%%

\chapter{DiscreteFunctions and Spaces}

%%%%%%%%%%%%%%%%%%%%%%%%%%%%%%%%%%%%%%%%%%%%%%%%%%%%%%%%%%%%%%%%


%%%%%%%%%%%%%%%%%%%%%%%%%%%%%%%%%%%%%%%%%%%%%%%%%%%%%%%%%%%%%%%%

\chapter{Operators}

%%%%%%%%%%%%%%%%%%%%%%%%%%%%%%%%%%%%%%%%%%%%%%%%%%%%%%%%%%%%%%%%


%%%%%%%%%%%%%%%%%%%%%%%%%%%%%%%%%%%%%%%%%%%%%%%%%%%%%%%%%%%%%%%%

\chapter{Passes}

%%%%%%%%%%%%%%%%%%%%%%%%%%%%%%%%%%%%%%%%%%%%%%%%%%%%%%%%%%%%%%%%

\section{General concept}

\section{An example}


%%%%%%%%%%%%%%%%%%%%%%%%%%%%%%%%%%%%%%%%%%%%%%%%%%%%%%%%%%%%%%%%

\chapter{Numerical methods}

%%%%%%%%%%%%%%%%%%%%%%%%%%%%%%%%%%%%%%%%%%%%%%%%%%%%%%%%%%%%%%%%

\section{Laplace equation, implicit computation}

\section{The finite volume method}


\section{The LDG method}


%%%%%%%%%%%%%%%%%%%%%%%%%%%%%%%%%%%%%%%%%%%%%%%%%%%%%%%%%%%%%%%%

\chapter{Solvers}

%%%%%%%%%%%%%%%%%%%%%%%%%%%%%%%%%%%%%%%%%%%%%%%%%%%%%%%%%%%%%%%%

\section{Forward Euler}


\section{CD method}


\section{IMEX method}


